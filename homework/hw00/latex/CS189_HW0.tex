%%%%% Don't Make Changes Below Here %%%%%
\documentclass{article}\usepackage[utf8]{inputenc}\usepackage[margin=0.4cm,top=0.4cm,bottom=0.4cm]{geometry}\usepackage[usenames,dvipsnames,svgnames,table]{xcolor}\usepackage{calligra}\usepackage{tikz}\usetikzlibrary{matrix,fit,chains,calc,scopes}\usepackage{tcolorbox}\tcbuselibrary{skins}\tcbset{Baystyle/.style={sharp corners,enhanced,boxrule=6pt,colframe=Aquamarine,height=\textheight,width=\textwidth,borderline={8pt}{-11pt}{},}}\usepackage{amsmath,amssymb,amsthm,tikz,tkz-graph,color,chngpage,soul,hyperref,csquotes,graphicx,floatrow}\newcommand*{\QEDB}{\hfill\ensuremath{\square}}\newtheorem*{prop}{Proposition}\renewcommand{\theenumi}{\alph{enumi}}\usepackage[shortlabels]{enumitem}\usetikzlibrary{matrix,calc}\MakeOuterQuote{"}\newtheorem{theorem}{Theorem} \usetikzlibrary{shapes} \usepackage{lipsum}\usepackage{tabularx,ragged2e,booktabs,caption}\tcbuselibrary{breakable}\newenvironment{yframed}{\begin{tcolorbox}[breakable,colback=gray!3,title after break={\textit{\color{red}Solution (cont.)}},colbacktitle=gray!3, coltitle=black,titlerule=-1pt] }{\end{tcolorbox}}\newtcolorbox{mybox}{colback=black!15!white, colframe=white,arc=12pt}\newtcolorbox{myboxot}{colback=green!15!white, colframe=white,arc=12pt,width=110pt, height=27pt}\newtcbox{\mylib}{enhanced,boxrule=0pt,top=0mm,bottom=0mm,right=0mm,left=4mm,arc=4pt,boxsep=9pt,before upper={\vphantom{dlg}},colframe=green!50!black,coltext=green!25!black,colback=green!10!white,overlay={\begin{tcbclipinterior}\fill[green!75!blue!50!white] (frame.south west)rectangle node[text=white,font=\sffamily\bfseries\tiny,rotate=90] {Problem} ([xshift=4mm]frame.north west);\end{tcbclipinterior}}}\newtcbox{\mylibot}{enhanced,boxrule=0pt,top=0mm,bottom=0mm,right=0mm,arc=4pt,boxsep=9pt,before upper={\vphantom{dlg}},colframe=green!50!black,coltext=green!25!black,colback=green!10!white,overlay={\begin{tcbclipinterior}\fill[red!75!blue!50!white] (frame.south west)rectangle node[text=white,font=\sffamily\bfseries\tiny,rotate=90] {Other} ([xshift=4mm]frame.north west);\end{tcbclipinterior}}}
\def\Title{\begin{tcolorbox}[Baystyle,]{\begin{center}\vspace*{0.14\textheight}
{\rule{\textwidth}{1.6pt}\vspace*{-\baselineskip}\vspace*{2pt}}
\rule{\textwidth}{0.4pt}\\[0.2\baselineskip]{\fontsize{45}{45}\scshape CS 189: Introduction to \\[-0.3\baselineskip] Machine Learning \\[0.2\baselineskip] \calligra Fall 2017 \\[0.2\baselineskip]}
{\rule{\textwidth}{0.4pt}\vspace*{-\baselineskip}\vspace{3.2pt}}
\rule{\textwidth}{1.6pt}\\[\baselineskip]\vspace{0.05\textheight}{{\fontsize{45}{45}\scshape$\bullet$\\ {Homework 0}\\\vspace*{0.01\textheight} }{{\fontsize{18}{18}\scshape{Due on Friday, August 25th, 2017 at 10 p.m.\\}}}\fontsize{45}{45}\scshape$\bullet$  \\}\vspace*{0.1\textheight}{\fontsize{12}{12}\calligra Solutions by\\}{\fontsize{28}{28}\scshape \Name \\}\vspace*{0.01\textheight}{\fontsize{12}{12}\scshape \SID} \\\vspace*{0.05\textheight}\end{center}}\end{tcolorbox}\newgeometry{margin=0.75in}}\def\BeginSolution{\begin{yframed}}\def\EndSolution{\end{yframed}}
\renewcommand{\baselinestretch}{1.25}
\newcommand{\circled[1]}{\tikz[baseline=(char.base)]{%
            \node[shape=circle,draw,inner sep=3pt] (char) {#1};}}
\newcommand{\chosen[1]}[black,fill=black]{\tikz[baseline=(char.base)]{%
            \node[shape=circle,draw,inner sep=3pt] (char) {#1};}}
%%%%% Don't Make Changes Above Here %%%%%

%%%%% Template Begins Here %%%%%

\def\Name{NAME}  % Your name
\def\SID{12345678}  % Your student ID number


\pagestyle{empty}
\begin{document}
\Title
%%%% Problem 1 Starts Here %%%%
\vspace{-2mm}\noindent\begin{mybox}{\begin{center}\textbf{\color{black}Problem 1: Getting Started}\end{center}}\end{mybox}

\begin{enumerate}[1.]
\item Before you start your homework, write down your team. Who else did you work with on this homework? List names and email addresses. In case of course events, just describe the group. How did you work on this homework? Any comments about the homework?
\BeginSolution % 1.1
%%%%YOUR SOLUTION HERE%%%%
\EndSolution
\item Please copy the following statement and sign next to it:\\
\textit{I certify that all solutions are entirely in my words and that I have not looked at another student's solutions. I have credited all external sources in this write up.}
\BeginSolution % 1.2
%%%%YOUR SOLUTION HERE%%%%
\includegraphics[width=\textwidth]{signature}
\EndSolution
\item How many hours did the homework take you to finish?\\ % 1.3
\textcircled{} 1 \hspace{10mm}\textcircled{} 2 \hspace{10mm}\textcircled{} 3 \hspace{10mm}\textcircled{} 4 \hspace{10mm}\textcircled{} 5 \hspace{10mm}\textcircled{} 6 \hspace{10mm}\textcircled{} 7 \hspace{10mm}\textcircled{} 8 \hspace{10mm}\textcircled{} 9 \hspace{10mm}\textcircled{$\checkmark$} 10\texttt{+}
\end{enumerate}

%%%% Problem 1 Ends Here %%%%
\clearpage

%%%% Problem 2 Starts Here %%%%
\vspace{-2mm}\noindent\begin{mybox}{\begin{center}\textbf{\color{black}Problem 2: Sample Submission}\end{center}}\end{mybox}

\noindent Please submit a plain text file to the Gradescope programming assignment "Homework 0 Test Set":

\begin{enumerate}[1.]
\item Containing 5 rows, where each row has only one value "1".
\item No spaces or miscellaneous characters.
\item Name it "submission.txt".
\end{enumerate}

%%%% Problem 2 Ends Here %%%%
\clearpage

%%%% Problem 3 Starts Here %%%%
\vspace{-2mm}\noindent\begin{mybox}{\begin{center}\textbf{\color{black}Problem 3: Eigendecomposition Review}\end{center}}\end{mybox}

\noindent Compute eigenvectors and eigenvalues for the following matrix. Show your work.
\begin{center}
$\begin{bmatrix}
1 & 3 \\
3 & 1
\end{bmatrix}$
\end{center}
\BeginSolution % 3
%%%YOUR SOLUTION HERE%%%%
\EndSolution
%%%% Problem 3 Ends Here %%%%
\clearpage

%%%% Problem 4 Starts Here %%%%
\vspace{-2mm}\noindent\begin{mybox}{\begin{center}\textbf{\color{black}Problem 4: Linear Regression and Adversarial Noise}\end{center}}\end{mybox}


\noindent Suppose we have training data consisting of $n$ points $(x_i, y_i)$, which we have modeled as coming from $y_i = ax_i +b$. We will do standard linear ordinary least-squares regression on the data to recover estimates for $a$ and $b$. Say that $y_i$ are actually coming from $y_i = ax_i+b+\varepsilon_i$, for some unknown disturbance process $\varepsilon_i$.
\begin{enumerate}[1.]
\item Can an adversary force the linear regression to recover any desired $a$, $b$ by setting exactly $1$ of the $\varepsilon_i$ to be a selected non-zero value?
\BeginSolution % 4 (1)
%%%YOUR SOLUTION HERE%%%
\EndSolution
\item What if the adversary sets $2$ of the $\varepsilon_i$?
\BeginSolution % 4 (2)
%%%YOUR SOLUTION HERE%%%
\EndSolution
\item How many does the adversary need to change and how would it do it?
\BeginSolution % 4 (3)
%%%YOUR SOLUTION HERE%%%
\EndSolution
\end{enumerate}
%%%% Problem 4 Ends Here %%%%
\clearpage

%%%% Problem 5 Starts Here %%%%
\vspace{-2mm}\noindent\begin{mybox}{\begin{center}\textbf{\color{black}Problem 5: Your Own Question}\end{center}}\end{mybox}

\noindent Write your own question, and provide a thorough solution.
\vspace{2mm}
\BeginSolution % 5 (a)
\textbf{\color{black} Question}
\vspace{2mm}\\
%%%YOUR QUESTION HERE%%%
\\\textbf{\color{red} Solution}
\vspace{2mm}
%%%YOUR SOLUTION HERE%%%
\EndSolution
%%%% Problem 5 Ends Here %%%%
\clearpage

\end{document}
%%%%% Template Ends Here %%%%%