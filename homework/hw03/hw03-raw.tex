\Question{Getting Started}

You may typeset your homework in latex or submit neatly handwritten and scanned solutions. Please make sure to start each question on a new page, as grading (with Gradescope) is much easier that way! Deliverables:

\begin{enumerate}
  \item Submit a PDF of your writeup to assignment on Gradescope, ``HW[n] Write-Up"
  \item Submit all code needed to reproduce your results, ``HW[n] Code".
  \item Submit your test set evaluation results, ``HW[n] Test Set".
\end{enumerate}

After you've submitted your homework, be sure to watch out for the self-grade form.

\begin{Parts}

\Part Before you start your homework, write down your team. Who else did you work with on this homework? List names and email addresses. In case of course events, just describe the group. How did you work on this homework? Any comments about the homework?

\vspace{15pt}
\framebox(465, 75){}

\Part Please copy the following statement and sign next to it:

\textit{I certify that all solutions are entirely in my words and that I have not looked at another student's solutions. I have credited all external sources in this write up.}

\vspace{15pt}
\framebox(465, 75){}

\end{Parts}

\pagebreak

Throughout the homework, note that \\
$Z \in \mathbb{R}$ implies that $Z$ is a scalar; \\
$Z \in \mathbb{R}^d$ is equivalent to saying that we have dimensions $\underbrace{Z}_{d \times 1}$; \\
$Z \in \mathbb{R}^{n \times d}$ is equivalent to saying that $Z$ is a matrix with dimensions $\underbrace{Z}_{n \times d}$. \\
When we say that $Z, Z' \in \mathbb{R}$, we mean that $Z \in \mathbb{R}$ and $Z' \in \mathbb{R}$.

\Question{Probabilistic Model of Linear Regression}
Both ordinary least squares and ridge regression have interpretations
from a probabilistic standpoint. In particular, assuming a generative
model for our data and a particular noise distribution, we will derive
least squares and ridge regression as the maximum likelihood and
maximum a-posteriori parameter estimates, respectively. This problem
will walk you through a few steps to do that. (Along with some side
digressions to make sure you get a better intuition for ML and MAP
estimation.)  

\begin{Parts}
\Part Assume that $X$ and $Y$ are both one-dimensional random
variables, i.e. $X, Y \in \mathbb{R}$. Assume an affine model between
$X$ and $Y$: $Y=Xw+b+Z$, where $w, b \in \mathbb{R}$, and $Z \sim
N(0,1)$ is a standard normal (Gaussian) random variable. Assume $w, b$ are not random. {\bf What is the conditional distribution of $Y$ given $X$?}



\Part Given $n$ points of training data $\{(X_1,Y_1),(X_2,Y_2),\ldots, (X_n,Y_n)\}$ generated in
an iid fashion by the probabilistic setting in the previous part, {\bf
  derive the maximum likelihood estimator for $w,b$ from this training
  data.} 




\Part Now, we are going to change the probabilistic model slightly in
two ways. There is no more $b$ and $Z$ has a different
distribution. So assume the linear model between $X$ and $Y$ is given
by $Y=Xw+Z$, where $Z \sim U[-0.5,0.5]$ is a continuous random
variable uniformly distributed between $-0.5$ and $0.5$. Assume $w$ is
not random. {\bf What is the conditional distribution of $Y$ given
  $X$?} 



\Part Given $n$ points of training data $\{(X_1,Y_1),(X_2,Y_2),\cdots,
(X_n,Y_n)\}$ generated in an iid fashion in the setting of the
previous part, {\bf derive the maximum likelihood estimator of $w$.}
(Note that this estimate may not be unique; you may report a set of values, or feel free to break ties in whatever fashion and report one value within this set. )



\Part Take the model $Y=Xw^*+b^*+Z$, where $Z \sim U[-0.5,0.5]$ is a continuous random
variable uniformly distributed between $-0.5$ and $0.5$. {\bf Use a computer
  to simulate $n$ training samples $\{(X_1,Y_1),(X_2,Y_2),\cdots,
(X_n,Y_n)\}$ and illustrate what the likelihood of the data looks like
  in the $(w,b)$ model parameter space after $n=5, 25, 125, 625$
  training samples. Qualitatively describe what is happening as $n$
  gets large?}  

(Note that you have considerable design freedom in this problem
part. You get to choose how you draw the $X_i$ as well as what true
value $(w^*,b^*)$ you want to illustrate. You have total freedom in
using python libraries for this problem part. No restrictions.) 


\Part (One-dimensional Ridge Regression) Now, let us return to the
case of Gaussian noise. Given $n$ points of training data
$\{(X_1,Y_1),(X_2,Y_2),\cdots, (X_n,Y_n)\}$ generated according to
$Y_i=X_i W+Z_i$, where $Z_i \sim N(0,1)$ are iid standard normal random
variables. Assume $W \sim N(0,\sigma^2)$ is also a standard normal and
is independent of both the $Z_i$ and the $X_i$.
{\bf Use Bayes' Theorem to derive the posterior distribution of $W$
  given the training data. What is the mean of the posterior distribution of $W$ given the data?}




\Part Consider $n$ training
data points $\{(X_1,Y_1),(X_2,Y_2),\cdots, (X_n,Y_n)\}$ generated
according to $Y_i=w^TX_i+Z_i$ where $Y_i\in\mathbb{R},w,X_i
\in\mathbb{R}^d$ with $w$ fixed, and $Z_i \sim N(0,1)$ iid standard
normal random variables.  {\bf Derive why the maximum likelihood estimator
 for $w$ is the solution to a least squares problem.} 





\Part (Multi-dimensional ridge regression) Consider the setup of the
previous part: $Y_i=W^TX_i+Z_i$, where $Y_i\in\mathbb{R},W,X_i
\in\mathbb{R}^d$, and $Z_i \sim N(0,1)$ iid standard
normal random variables. Now, however, we have a prior for the vector $W$ (now a
random variable): $W_j\sim N(0,\sigma^2)$ are iid for $j=1,2,\ldots,d$.  {\bf Derive the posterior
  distribution of $W$ given all the $X_i,Y_i$ pairs. What is the mean of the
  posterior distribution of the vector $W$?} 



\Part Consider $d=2$ and the setting of the previous part. {\bf Use a
  computer to simulate and illustrate what the a-posteriori
  probability looks like for the $W$ model parameter space after $n=5,
  25, 125$ training samples for different values of
  $\sigma^2$. } 

(Note that you have considerable design freedom in this problem
part. You have total freedom in using python libraries for this
problem part. No restrictions. Either contour plots or 3d plots of the
a-posteriori probability are fine. You don't have to label the
contours with values if the story is clear from the plots visually.)  


\end{Parts}
\Question{Simple Bias-Variance Tradeoff}

Consider a random variable $X$, which has unknown mean $\mu$ and
unknown variance $\sigma^2$. Given $n$ iid realizations of training samples $X_1=x_1,
X_2=x_2, \ldots, X_n=x_n$ from the random variable, we wish to
estimate the mean of $X$. We will call our estimate the random variable
$\hat{X}$ with mean $\hat{\mu}$. There are a few ways we can estimate
$\mu$ given the realizations of the $n$ samples:
\begin{enumerate}
	\item Average the $n$ samples: $\frac{x_1+x_2+\ldots+x_n}{n}$.
	\item Average the $n$ samples and one sample of $0$: $\frac{x_1+x_2+\ldots+x_n}{n+1}$.
	\item Average the $n$ samples and $n_0$ samples of $0$: $\frac{x_1+x_2+\ldots+x_n}{n+n_0}$.
	\item Ignore the samples: just return $0$.
\end{enumerate}

In the parts of this question, we will measure the \emph{bias} and \emph{variance} of each of our estimators. The \emph{bias} is defined as $$E[\hat{X} - \mu]$$ and the $\emph{variance}$ is defined as $$\text{Var}[\hat{X}].$$

\begin{Parts}
\Part \textbf{What is the bias of each of the four estimators above?}



\Part \textbf{What is the variance of each of the four estimators
  above?}



\Part Hopefully you see that there is a trade-off. As we add more
copies of $0$ to our estimator, the bias increases and the variance
decreases. It is a common mistake to assume that an unbiased estimator
is always ``best.'' Let's explore this a bit further. Define the
\emph{expected total error} as the sum of the variance and the square
of the bias. \textbf{Compute the expected total error for each of the estimators above.}



\Part \textbf{Argue that all four estimators are really just special cases of the third estimator (with the hyperparameter $n_0$)}.



\Part Say that $n_0 = \alpha n$. \textbf{Find the setting for $\alpha$
  that would minimize the expected total error, assuming you secretly
  knew $\mu$ and $\sigma$.} Your answer will depend on $\sigma$, $\mu$, and $n$.



\Part For this part, let's assume that we had some reason to believe that $\mu$ \emph{should be small} (close to $0$) and $\sigma$ \emph{should be large}. In this case, \textbf{what happens to the expression in the previous part?}


\Part In the previous part, we assumed there was reason to believe that $\mu$ \emph{should be small}. Now let's assume that we have reason to believe that $\mu$ is not necessarily small, but \emph{should be close to some fixed value} $\mu_0$. \textbf{In terms of $X$ and $\mu_0$, how can we define a new random variable $X'$ such that $X'$ is expected to have a small mean?}



\Part Draw a connection between $\alpha$ in this problem and the
regularization parameter $\lambda$ in the ridge-regression version of least-squares. \textbf{What does this problem suggest
  about choosing a regularization coefficient and handling our
  data-sets so that regularization is most effective}? This is an
open-ended question, so do not get too hung up on it. 



\end{Parts}
\Question{Estimation in linear regression}

In linear regression, we estimate a vector $y \in \mathbb{R}^n$ by
using the columns of a feature matrix $A \in \mathbb{R}^{n \times
  d}$. Assume that the number of training samples $n \geq d$ and that
$A$ has full column rank. Let us now show how well we can estimate the
underlying model as the number of training samples grows.

Assume that the true underlying model for our noisy training
observations is given by $Y = \underbrace{Ax^*}_{y^*} + W$, with $W
\in \mathbb{R}^n$ having iid $W_j \sim \mathcal{N}(0,\sigma^2)$
representing the random noise in the observation $Y$. Here, the $x^*
\in \mathbb{R}^d $ is something arbitrary and not random. After obtaining $\hat{X} = \arg\min_x \| Y - A
x\|_2^2$, we would like to bound the error $\| A \hat{X} - y^*
\|_2^2$, which is the prediction error incurred based on the specific
$n$ training samples we obtained. 

Initially, it might seem that getting a good estimate of this
prediction error is hopeless since the training data's $A$ matrix is
involved in some complicated way in the estimate. The point of this
problem is to show you that this is not the case and we can actually
understand this. 

\begin{Parts}
\Part Using the standard closed form solution to the ordinary least squares problem, {\bf show that}
$$\| A \hat{X} - y^* \|_2^2 = \| A (A^\top A)^{-1} A^\top W \|_2^2.$$



\Part Given the \emph{reduced} singular value decomposition $A = U
\Sigma V^\top$ (with $U \in \mathbb{R}^{n \times d}$, $\Sigma \in
\mathbb{R}^{d \times d}$, and $V \in \mathbb{R}^{d \times d}$. The
reduced SVD just ignores all the singular vectors that correspond to
directions that are in the nullspace of $A^\top$ since $A$ is a tall
skinny matrix.), {\bf show that we have}
$$\| A \hat{X} - y^* \|_2^2 = \| U^\top W \|_2^2.$$

(Hint: Recall that the standard Euclidean $\ell_2$ norm is unitarily invariant.)



\Part If $W_0$ is a vector with i.i.d standard Gaussians, recall that
$a^\top W_0 \sim \mathcal{N}(0, \|a\|_2^2)$.  {\bf Use this fact to show that}
$$\frac{1}{n} \EE \left[ \| A \hat{X} - y^* \|_2^2 \right] = \sigma^2 \frac{d}{n}.$$ 

Notice that this kind of scaling is precisely how the average squared error for simple
averaging scales when we have multiple iid samples of a random
variable and we want to estimate its mean. This is why we often think
about ordinary least squares as just being a kind of generalized
averaging. However, the dimension of the parameters being estimated
also matters.



\Part Let us now try to answer another basic question. Let us say we
have a true linear model, but we try to estimate it using a polynomial of degree $D$. Clearly, this is overkill, but what do we lose?

Given scalar samples $\{x_i, Y_i\}_{i=1}^n$, let us say that
the underlying model is a noisy linear model, i.e. $Y_i = m x_i + c +
W_i$. Let us, however, perform polynomial regression: we form the
matrix $A$ by using $D+1$ polynomial features (including the constant)
of the \emph{distinct} sampling points $\{x_i\}_{i=1}^n$. {\bf How many samples $n$ do
  we need to ensure that our average squared error is bounded by $\epsilon$?} Your answer should be expressed as a function of $D$, $\sigma^2$, and $\epsilon$.

{\bf Conclude that as we increase model complexity, we require a
  proportionally larger number of samples for accurate prediction.}

At this point, you are encouraged to take a fresh look at the
discussion worksheet which is, in effect, a continuation of this
problem to understand the ramifications of the bias/variance tradeoff
in the context of ordinary least-squares regression. 



\Part Try the above problem out by yourself, by setting $m = 1$, $c =
1$, and sample $n$ points $\{x_i\}_{i=1}^n$ uniformly from the
interval $[-1,1]$. Generate $Y_i = m x_i + c + W_i$ with $W_i$
representing standard Gaussian noise. {\bf Fit a $D$ degree polynomial
  to this data and show how the average error $\frac{1}{n} \| A
  \hat{X} - y^* \|_2^2$ scales as a function of both $D$ and $n$.} You
may show separate plots for the two scalings. It may also be helpful
to average over multiple realizations of the noise (or to plot point
clouds) so that you obtain smooth curves. 

(For this part, any and all python libraries are allowed.)

\end{Parts}
\Question{Robotic Learning of Controls from Demonstrations and Images}

Huey, a home robot, is learning to retrieve objects from a cupboard, as shown in Fig.~\ref{fig:robot}. The goal is to push obstacle objects out of the way to expose a goal object.  Huey's robot trainer, Anne, provides demonstrations via tele-operation. 

During a demonstration, Huey records the RGB images of the scene for each timestep, $x_0,x_1,...,x_{T}$, where $x_i \in \mathbb{R}^{30x30x3}$ and the controls for his mobile base, $u_0,u_1,\ldots,u_{T}$, where $u_i \in \mathbb{R}^3$. The controls correspond to making small changes in the 3D pose (i.e. translation and rotation) of his body. Examples of the data are shown in the figure. 

Under an assumption (sometimes called the Markovian assumption) that all that matters for the current control is the current image, Huey can try to learn a linear \emph{policy} $\pi$ (where $\pi \in \mathbb{R}^{2700x3}$) which linearly maps image states to controls (i.e. $\pi^\top x =u$). We will now explore how Huey can recover this policy using linear regression. Note please use {\bf numpy} and {\bf numpy.linalg} to complete this assignment. 

\begin{figure}[h!]
    \begin{center}
    \includegraphics[scale=.05]{src/problems/linear_regression/robot.pdf}
    \caption{A) Huey trying to retrieve a mustard bottle. An example RGB image of the workspace taken from his head mounted camera is shown in the orange box. The angle of the view gives Huey and eye-in-hand perspective of the cupboard he is reaching into. B) A scatter plot of the 3D control vectors, or $u$ labels. Notice that each coordinate of the label lies within the range of $[-1,1]$ for the change in position. Example images, or states $x$, are shown for some of the corresponding control points. The correspondence is indicated by the blue lines.  } \label{fig:robot}
    \end{center}
\end{figure}

\begin{Parts}
\Part Load the $n$ training examples from x\_train.p and compose the matrix $X$, where $X \in \mathbb{R}^{nx2700}$. Note, you will need to flatten the images to reduce them to a single vector. The flattened image vector will be denoted by $\bar{x}$ (where $\bar{x} \in \mathbb{R}^{2700x1}$). Next, load the $n$ examples from y\_train.p and compose the matrix $U$, where $U \in \mathbb{R}^{nx3}$. Try to perform ordinary least squares to solve:  

$$\min_{\pi} \|X\pi-U \|^F_2$$

to learn the \emph{policy} $\pi \in \mathbb{R}^{2700 \times 3}$. {\bf Report what happens as you attempt to do this and explain why.}



\Part Now try to perform ridge regression:

$$\underset{\pi}{\mbox{min}} \: ||X\pi-U||^2_2 + \lambda ||\pi||^2_2$$

on the dataset for regularization values $\lambda = \lbrace 0.1,1.0,10,100,1000 \rbrace$. Measure the average squared Euclidean distance for the accuracy of the policy on the training data:

$$\frac{1}{n}\sum_{i =0 }^{n-1} ||\bar{x}_i^T \pi - u_i||^2_2$$ 

{\bf Report the training error results for each value of $\lambda$}.



\Part Next, we are going to try and improve the performance by standardizing the states. For each pixel value in each data point, $x$, perform the following operation: 

	$$x = \frac{x}{255} *2 - 1.$$
 Since we know the maximum pixel value is $255$, this rescales the data to be between $[-1,1]$ . {\bf Repeat the previous part and report the average squared training error for each value of $\lambda$}.




\Part To better understand how standardizing improved the loss function, we are going to evaluate the \emph{condition number} $k$ of the optimization, which is defined as

$$k = \frac{\sigma^2_{\mbox{max}}(X^TX+\lambda I)}{\sigma^2_{\mbox{min}}(X^TX+\lambda I)}$$
or the ratio of the maximum eigenvalue to the minimum eigenvalue of the relevant matrix. For the regularization value of $\lambda = 100$, {\bf report the condition number with the standardization technique applied and without}. 









\Part Finally, evaluate your learned \emph{policy} on the new validation data x\_test.p and y\_test.p for the different values of $\lambda$. {\bf Report the average squared Euclidean loss} and {\bf qualitatively explain how changing the values of $\lambda$ affects the performance in terms of bias and variance}. 




\end{Parts}
\Question{Your Own Question}

{\bf Write your own question, and provide a thorough solution.}

Writing your own problems is a very important way to really learn
material. The famous ``Bloom's Taxonomy'' that lists the levels of
learning is: Remember, Understand, Apply, Analyze, Evaluate, and
Create. Using what you know to create is the top-level. We rarely ask
you any HW questions about the lowest level of straight-up
remembering, expecting you to be able to do that yourself. (e.g. make
yourself flashcards) But we don't want the same to be true about the
highest level.

As a practical matter, having some practice at trying to create
problems helps you study for exams much better than simply counting on
solving existing practice problems. This is because thinking about how
to create an interesting problem forces you to really look at the
material from the perspective of those who are going to create the
exams. 

Besides, this is fun. If you want to make a boring problem, go
ahead. That is your prerogative. But it is more fun to really engage
with the material, discover something interesting, and then come up
with a problem that walks others down a journey that lets them share
your discovery. You don't have to achieve this every week. But unless
you try every week, it probably won't happen ever. 