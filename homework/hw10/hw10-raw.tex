\Question{Getting Started}

You may typeset your homework in latex or submit neatly handwritten and scanned solutions. Please make sure to start each question on a new page, as grading (with Gradescope) is much easier that way! Deliverables:

\begin{enumerate}
  \item Submit a PDF of your writeup to assignment on Gradescope, ``HW[n] Write-Up"
  \item Submit all code needed to reproduce your results, ``HW[n] Code".
  \item Submit your test set evaluation results, ``HW[n] Test Set".
\end{enumerate}

After you've submitted your homework, be sure to watch out for the self-grade form.

\begin{Parts}

\Part Before you start your homework, write down your team. Who else did you work with on this homework? List names and email addresses. In case of course events, just describe the group. How did you work on this homework? Any comments about the homework?

\vspace{15pt}
\framebox(465, 75){}

\Part Please copy the following statement and sign next to it:

\textit{I certify that all solutions are entirely in my words and that I have not looked at another student's solutions. I have credited all external sources in this write up.}

\vspace{15pt}
\framebox(465, 75){}

\end{Parts}

\pagebreak

\Question{Clip Loss}

In lecture, you saw the example of different loss functions like the
squared-error loss and the hinge-loss. This question explores a
different loss function. 

Let $S=\{(x_1,y_1), \ldots (x_n,y_n)\}$ be a set of $n$ points sampled
i.i.d.~from a distribution $\mathcal{D}$.  This is the training set
with $x_i \in \mathbb{R}^d$ being the features and $y_i \in \{-1,1\}$
being the labels.

We are thinking about a linear classifier that is going to look at the
sign of $w^T x$ to make a decision as to whether the label is $+1$ or
$-1$. 

Define the \emph{clip loss} of a linear classifier $w\in \mathbb{R}^d$ as
\[
	\operatorname{loss}(w^T x,y) = \operatorname{clip}(y w^T x)
\]
Where $\operatorname{clip}$ is the function
\[
\operatorname{clip}(z) = \begin{cases} 1 & \mbox{if}~z <0\\
		0 & \mbox{if}~z \geq 1\\
				1-z & \mbox{otherwise.}
		\end{cases}
\]
For any $d$-dimensional vector $w$, define the \emph{risk} of $w$ as
\[
	R[w] = \mathbb{E}_\mathcal{D}[\operatorname{loss}(w^T x,y)]\,,
\]
and the \emph{empirical risk} of $w$ as
\[
	R_S[w] = \frac{1}{n} \sum_{i=1}^n \operatorname{loss}(w^T x_i,y_i)\,.
\]

\begin{Parts}

\Part Draw the clip loss function. \textbf{Is the function $\operatorname{clip}$ convex?} Justify your answer.



\Part {\bf Interpret the loss function. Why is it reasonable to look
  at $y w^T x$?} 



\Part \textbf{Prove that if $R_S[w]=0$ and $\|w\|^2_2 < 1$, then the
  hyperplane defined by $w$ has a classification margin greater than 1 on this
  training set.} 



\Part \textbf{Prove that $\mathbb{E}_S[R_S[w]] = R[w]$.} Here, the
outer expectation is being taken over the randomly drawn training
set. 



\Part \textbf{Prove that $\operatorname{Var} (R_S[w]) \leq
  \frac{1}{n}$.} 



\Part \textbf{Is it possible to have an $S$ and $w$ such that $R_S[w]=0$, but $R[w]>0$?}  Justify your answer.



\end{Parts}
\Question{Gradient Descent Methods Convergence}

In this problem we will examine the convergence of different gradient
descent methods that we have seen so far. We aim to solve a linear
least squares problem on the dataset gradient\_descent\_data.mat. Remember that the linear least squares problem can be written as:

$$\min_w \frac{1}{2} || Aw - y||^2_2 = \min_w \frac{1}{2} \Sigma_{i=1}^n (A_i^\top w - y_i)^2 = \min_w \frac{1}{2} \sum_{i=1}^n \ell_i(w)$$.

Here $A_i^\top$ denotes the $i$th row of matrix $A$ and $\ell_i$ is the loss of the training example $i$. We implement a variant of SGD, called randomized Kaczmarz SGD as follows:

$$w^{t} = w^{t-1} - \alpha_J \nabla \ell_J(w^{t-1})$$

with learning rates $\alpha_j = \frac{1}{\|A_j^{\top}\|_2^2}$ for each
$j = 1, 2, \ldots, n$. Above, $J$ is a random index between $1$ and
$n$ such that the probability of choosing $J=j$ is given by $p(J=j) =
\frac{||A_{j}^{\top}||_2^2}{||A||_F^2}$ for $j \in \{1, \ldots
,n\}$. The superscript $t$ on the weights $w^t$ tells us which
iteration we are considering.  

For this problem, only use these python libraries: numpy, matplotlib, scipy.io

\begin{Parts}
\Part \textbf{Write code to apply a batch gradient descent algorithm
  10 times over all the samples with learning rate $\alpha = 0.05$.}
Initialize the weight vector with zeros. \textbf{Plot the mean square
  error as a function of number of iterations.} 


\Part \textbf{Write code to apply a standard stochastic gradient
  descent algorithm with minibatch size of one with fixed learning rate
  $\alpha = 0.05$.} Initialize the weight vector with
zeros. Do a number of iterations that corresponds to 10 times over the samples. \textbf{Plot the mean square error as a function of number of
  iterations. What happens if we increase the learning rate? Compare
  the results with previous part.} To be fair in our comparisons from
a computational perspective, consider each time we have processed $n$
samples to be the equivalent of one iteration in gradient descent. So
we have ``fractional'' iterations here. 



\Part Stochastic gradient descent looks at the training data in small
minibatches (or one at a time) instead of looking at the whole big
data set each time. Another thing that we could do is to look at the
{\em features} one at a time. This is sometimes called ``coordinate
descent'' in optimization because you work with one coordinate at a
time -- changing just the weight on that one feature. (Leaving all the
other weights unchanged for this iteration.) 

Do all three possibilities for coordinate descent for this
problem. In each case at each iteration, uniformly pick one of the
features and only adjust its weight during this iteration. During each
iteration: (1) Do a full least-squares solution, but assuming that the
only thing you can change is this single weight (think about how easy
least squares is when you only have a single unknown to fit). (2) Do a full gradient
descent, but assuming that only this one weight can be changed this
iteration. (3) Do stochastic gradient descent using a minibatch size
of one, but assuming that for this iteration, only this one weight can
be changed. 

\textbf{Write code to uniformly randomly subsample from the columns
  (the two features). Plot the mean square error as a function of
  number of iterations for each of these possibilities. Compare the
  results with the previous parts.} 



\Part \textbf{Write code to apply randomized Kaczmarz stochastic
  gradient descent. Plot the mean square training error as a function of number of iterations. Compare the results with the previous parts.}



\Part \textbf{Now add some nonlinearity to the data by adding $0.1 \times a_2^3$ to $y$, where $a_2$ is the second column of the data matrix $A$. Redo the previous parts and compare the results.}



\end{Parts}
\Question{Linear Methods on Fruits and Veggies}
In this problem, we will use the dataset of fruits and vegetables that
was collected in HW5 and HW6. The goal is to accurately classify the
produce in the image. Instead of operating on the raw pixel values, we
operate on extracted HSV histogram features from the image. HSV
histogram features extract the color spectrum of an image, so we
expect these features to serve well for distinguishing produce like
bananas from apples. Denote the input state $x \in \mathbb{R}^{729}$,
which is an HSV histogram generated from an RGB image with a fruit
centered in it. Each data point will have a corresponding class label,
which corresponds to their matching produce. Given 25 classes, we can
denote the label as $y \in \lbrace 0,...,24 \rbrace$. 

Better features would of course give better results, but we chose
color spectra for an initial problem for ease of interpretation.


Classification here is still a hard problem because the state space is
much larger than the amount of data we obtained in the class -- we are
trying to perform classification in a $729$ dimensional space with only a few hundred data points from each of the $25$ classes. In order to obtain higher accuracy, we will examine how to perform hyper-parameter optimization and dimensionality reduction. We will first build out each component and test on a smaller dataset of just 3 categories: apple, banana, eggplant. Then we will combine the components to perform a search over the entire dataset.


Note all python packages needed for the project, will be imported already. {\bf DO NOT import new Python libraries.}



\begin{Parts}
\Part Before we classify our data, we will study how to reduce the
dimensionality of our data. We will project some of the dataset into 2D to visualize how effective different dimensionality reduction procedures are. The first method we consider is a random projection, where a matrix is randomly created and the data is linearly projected along it. 

For random projections {\bf fill in the function $\mbox{get\_rand\_proj}$ in $\mbox{projection.py}$} to produce a matrix, $A \in \mathbb{R}^{2 \times 729}$ where each element $A_{ij}$ is sampled independently from a normal distribution (i.e. $A_{ij} \sim N(0,1)$). {\bf Run the code $\mbox{projection.py}$ and report the resulting plot of the projection.}







\Part We will next examine how well PCA performs as a
dimensionality-reduction tool. PCA projects the data into the subspace with the most variance, which is determined via the covariance matrix $\Sigma_{XX}$. We can compute the principal components via the singular value decomposition $U\Lambda U^T = \Sigma_{XX}$. Collecting the first two column vectors of $U$ in the matrix $U_2$, we can project our data as follows: 

$$\bar{x} = U_2^T x$$


 {\bf Fill in the function  $\mbox{get\_pca\_proj}$}. Note, you can use the functions imported frum $\mbox{utils}$ to compute the covariance matrix from data and perform mean subtraction.



\Part Finally, we will project our data into the Canonical Variates. In order to perform CCA, we must first turn our labels $y$ into a one-hot encoding vector $\bar{y} \in \lbrace 0,1 \rbrace^{J}$, where each element corresponds to the label. Note $J$ is the number of class labels, which is $J = 3$ for this part.   Next we need to compute the canonical correlation matrix $\Sigma_{XX}^{-\frac{1}{2}}\Sigma_{XY}\Sigma_{YY}^{-\frac{1}{2}}$ and compute the singular value decomposition $U\Lambda V^T = \Sigma_{XX}^{-\frac{1}{2}}\Sigma_{XY}\Sigma_{YY}^{-\frac{1}{2}}$. 

We can then project to the canonical variates by using the first $k$ columns in $U$, or $U_k$. The projection can be written as follows: 

$$\bar{x} = U_k^T \Sigma_{XX}^{-\frac{1}{2}} x$$

{\bf Fill in \mbox{get\_cca\_proj} and report the resulting plot for $k=2$}. Note you can use the given function $\mbox{create\_one\_hot\_label}$.




\Part We will now examine ways to perform classification using this
smaller projected space from CCA as our features. One technique is to regress to the class labels and then greedily choose the model's best guess. In this problem, we will use ridge regression to learn a mapping from the HSV histogram features to the one-hot encoding $\bar{y}$ described in the previous problem.  Solve the following Ridge Regression problem:

$$\underset{w}{\mbox{min}} \sum^{N}_{n=1} ||\bar{y} - w^Tx_n||_2^2 + \lambda ||w||^2_F$$

Then we will make predictions with the following function:

$$y = \underset{j\in{0,..,J-1}}{\mbox{argmax}} \: (w^Tx)_j$$

where $(w^Tx)_j$ considers the $j$-th coordinate of the predicted vector. {\bf Implement the skeleton class $\mbox{Ridge\_Model}$}. Note that you are allowed to use the imported SKlearn's Ridge Regression method. 

Once your model is implemented {\bf run $\mbox{linear\_classification.py}$}. It will output a confusion matrix, a matrix that compares the actual label to the predicted label of the model. The higher the numerical value on the diagonal, the higher the percentage of correct predictions made by the model, thus the better model. {\bf Report the Ridge Regression confusion matrix for the training data and validation data}.



\Part Instead of performing regression, we can potentially obtain better performance by using algorithms that explicitly model a classification procedure. LDA (Linear Discriminant Analysis) approaches the problem by assuming each $p(x | y =j)$ is a normal distribution with mean $\mu_j$ and covariance $\Sigma$. Notice that the covariance matrix is assumed to be the same for all the class labels. 

LDA works by fitting $\mu_j$ and $\Sigma$ on the dimensionality-reduced dataset. During prediction, the class with the highest likelihood is chosen. 

$$y = \underset{j\in{0,...,J-1}}{\mbox{argmax}} \: -(x-\mu_j)^T\Sigma^{-1}(x-\mu_j)$$

{\bf Fill in the class $\mbox{LDA\_Model}$}. Then {\bf run $\mbox{linear\_classification.py}$ and report the LDA confusion matrix for the training and validation data}.





 \Part LDA makes an assumption that all classes have the same covariance matrix. We can relax this assumption with QDA. In QDA, we will now parametrize each conditional distribution (still normal) by $\mu_j$ and $\Sigma_j$. The prediction function is then computed as 

 $$y = \underset{j\in{0,..,J-1}}{\mbox{argmax}} \: -(x-\mu_j)^T\Sigma_j^{-1}(x-\mu_j) - \log(|\Sigma_j|) $$


{\bf Fill in the class $\mbox{QDA\_Model}$}. Then {\bf run $\mbox{linear\_classification.py}$ and report the QDA confusion matrix for the training data and validation data}.

 

 \Part Finally, let us examine the Linear SVM, which fits a hyperplane
 to separate the dimensionality-reduced data. For this problem, you can use SKlearn's implementation. 
 {\bf Fill in the class $\mbox{SVM\_Model}$}. Then {\bf run $\mbox{linear\_classification.py}$ and report the Linear SVM confusion matrix for the training data and validation data}.




 \Part We will finally train on the full dataset and compare the different models.  We will perform a grid search over the following parameters:
 \begin{enumerate}
\item The regularization term $\lambda$ in Ridge Regression. 
\item The weighting on slack variables, $C$ in the linear SVM. 
\item The number of dimensions, $k$ we project to using CCA.
\end{enumerate} 

The file $\mbox{hyper\_search.py}$ contains the parameters that will
be swept over. If the code is correctly implemented in the previous
steps, this code should perform a sweep over all parameters and give
the best model. {\bf Run $\mbox{hyper\_search.py}$, report the model
  parameters chosen, report the plot of the models's validation error, and report the best model's confusion matrix for validation data.} WARNING: This can take up to an hour to run.






\end{Parts}

\Question{Your Own Question}

{\bf Write your own question, and provide a thorough solution.}

Writing your own problems is a very important way to really learn
material. The famous ``Bloom's Taxonomy'' that lists the levels of
learning is: Remember, Understand, Apply, Analyze, Evaluate, and
Create. Using what you know to create is the top-level. We rarely ask
you any HW questions about the lowest level of straight-up
remembering, expecting you to be able to do that yourself. (e.g. make
yourself flashcards) But we don't want the same to be true about the
highest level.

As a practical matter, having some practice at trying to create
problems helps you study for exams much better than simply counting on
solving existing practice problems. This is because thinking about how
to create an interesting problem forces you to really look at the
material from the perspective of those who are going to create the
exams. 

Besides, this is fun. If you want to make a boring problem, go
ahead. That is your prerogative. But it is more fun to really engage
with the material, discover something interesting, and then come up
with a problem that walks others down a journey that lets them share
your discovery. You don't have to achieve this every week. But unless
you try every week, it probably won't happen ever. 